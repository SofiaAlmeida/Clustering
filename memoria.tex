%%%
% Plantilla de Memoria
% Modificación de una plantilla de Latex de Nicolas Diaz para adaptarla 
% al castellano y a las necesidades de escribir informática y matemáticas.
%
% Editada por: Mario Román
%
% License:
% CC BY-NC-SA 3.0 (http://creativecommons.org/licenses/by-nc-sa/3.0/)
%%%

%%%%%%%%%%%%%%%%%%%%%%%%%%%%%%%%%%%%%%%%%
% Thin Sectioned Essay
% LaTeX Template
% Version 1.0 (3/8/13)
%
% This template has been downloaded from:
% http://www.LaTeXTemplates.com
%
% Original Author:
% Nicolas Diaz (nsdiaz@uc.cl) with extensive modifications by:
% Vel (vel@latextemplates.com)
%
% License:
% CC BY-NC-SA 3.0 (http://creativecommons.org/licenses/by-nc-sa/3.0/)
%
%%%%%%%%%%%%%%%%%%%%%%%%%%%%%%%%%%%%%%%%%

%----------------------------------------------------------------------------------------
%	PAQUETES Y CONFIGURACIÓN DEL DOCUMENTO
%----------------------------------------------------------------------------------------

%%% Configuración del papel.
% microtype: Tipografía.
% mathpazo: Usa la fuente Palatino.
\documentclass[a4paper, 20pt]{article}
\usepackage[a4paper,margin=1in]{geometry}
\usepackage[protrusion=true,expansion=true]{microtype}
\usepackage{mathpazo}

% Indentación de párrafos para Palatino
\setlength{\parindent}{0pt}
  \parskip=8pt
\linespread{1.05} % Change line spacing here, Palatino benefits from a slight increase by default


%%% Castellano.
% noquoting: Permite uso de comillas no españolas.
% lcroman: Permite la enumeración con numerales romanos en minúscula.
% fontenc: Usa la fuente completa para que pueda copiarse correctamente del pdf.
\usepackage[spanish,es-noquoting,es-lcroman,es-tabla,,es-nodecimaldot]{babel}
\usepackage[utf8]{inputenc}
\usepackage{fontenc}
\selectlanguage{spanish}

%%% Matemáticas
\usepackage{amsmath}

%%% Gráficos
\usepackage{graphicx} % Required for including pictures
\usepackage{wrapfig} % Allows in-line images
\graphicspath{{./fig/}}
\usepackage[usexcolor=false, inkscape=true]{svg} % Required for including svg
\svgpath{{./fig/}}
\usepackage[usenames,dvipsnames]{color} % Coloring code



%%% Pseudocódigo
\usepackage{algorithmicx}
\usepackage[ruled]{algorithm}
\usepackage{algpseudocode}

\newcommand{\alg}{\texttt{algorithmicx}}
\newcommand{\old}{\texttt{algorithmic}}
\newcommand{\euk}{Euclid}
\newcommand\ASTART{\bigskip\noindent\begin{minipage}[b]{0.5\linewidth}}
\newcommand\ACONTINUE{\end{minipage}\begin{minipage}[b]{0.5\linewidth}}
\newcommand\AENDSKIP{\end{minipage}\bigskip}
\newcommand\AEND{\end{minipage}}

%%% Código
\usepackage{listings}

%%% Tablas
\usepackage{tabularx}
\usepackage{float}
\usepackage{adjustbox}
\usepackage{booktabs}
\usepackage{array}
\newcolumntype{L}[1]{>{\raggedright\let\newline\\\arraybackslash\hspace{0pt}}m{#1}}
\newcolumntype{C}[1]{>{\centering\let\newline\\\arraybackslash\hspace{0pt}}m{#1}}
\newcolumntype{R}[1]{>{\raggedleft\let\newline\\\arraybackslash\hspace{0pt}}m{#1}}

% Enlaces y colores
\usepackage{hyperref}
\usepackage{xcolor}
\definecolor{webgreen}{rgb}{0,0.5,0}
\hypersetup{
  colorlinks=true,
  citecolor=RoyalBlue,
  urlcolor=RoyalBlue,
  linkcolor=RoyalBlue
}

%%% Bibliografía
%\usepackage[backend=biber]{biblatex}
%\DefineBibliographyStrings{spanish}{
%  urlseen = {Último acceso}
%}
%\addbibresource{IN-P2.bib}

%----------------------------------------------------------------------------------------
%	TÍTULO
%----------------------------------------------------------------------------------------
% Configuraciones para el título.
% El título no debe editarse aquí.
\renewcommand{\maketitle}{
  \begin{flushright} % Right align
  
  {\LARGE\@title} % Increase the font size of the title
  
  \vspace{50pt} % Some vertical space between the title and author name
  
  {\large\@author} % Author name
  \\\@date % Date
  \vspace{40pt} % Some vertical space between the author block and abstract
  \end{flushright}
}

%% Título
\title{\textbf{Título}\\ % Title
Subtítulo} % Subtitle

\author{\textsc{Autor1,\\Autor2} % Author
\\{\textit{Universidad de Granada}}} % Institution

\date{\today} % Date

%-----------------------------------------------------------------------------------------
%	DOCUMENTO
%-----------------------------------------------------------------------------------------

\begin{document}

%-----------------------------------------------------------------------------------------
%	TITLE PAGE
%-----------------------------------------------------------------------------------------

\begin{titlepage} % Suppresses displaying the page number on the title page and the subsequent page counts as page 1
	
	\raggedleft % Right align the title page
	
	\rule{1pt}{\textheight} % Vertical line
	\hspace{0.05\textwidth} % Whitespace between the vertical line and title page text
	\parbox[b]{0.8\textwidth}{ % Paragraph box for holding the title page text, adjust the width to move the title page left or right on the page
		
		{\Huge\bfseries \textit{Clustering}}\\[2\baselineskip] % Title
		{\large\textit{Curso 2019/2020}}\\[4\baselineskip] % Subtitle or further description
		{\Large\textsc{Sofía Almeida Bruno}\\\textsc{Daniel Bolaños Martínez}\\\textsc{José María Borrás Serrano}\\\textsc{Fernando de la Hoz Moreno}\\\textsc{Pedro Manuel Flores Crespo}\\\textsc{María Victoria Granados Pozo}} % Author name, lower case for consistent small caps
		
		\vspace{0.4\textheight} % Whitespace between the title block and the publisher
		
		{\noindent }\\[\baselineskip] % Publisher and logo
	}

\end{titlepage}

%% Resumen (Descomentar para usarlo)
%\renewcommand{\abstractname}{Resumen} % Uncomment to change the name of the abstract to something else
%\begin{abstract}
% Resumen aquí
%\end{abstract}

%% Palabras clave
%\hspace*{3,6mm}\textit{Keywords:} lorem , ipsum , dolor , sit amet , lectus % Keywords
%\vspace{30pt} % Some vertical space between the abstract and first section


%% Índice
{\parskip=2pt
  \tableofcontents
}
\pagebreak
% TODO:
% - Pasar a textit: clustering
% TODO:
% - Referenciar referencias...

\section{Introducción}
\section{Medidas}
\section{Métodos de agrupamiento}
\subsection{Jerárquicos}
\subsection{No jerárquicos}
\section{Número de clústeres}
\section{Parte práctica}
%%%%%%%%%%%%%%%%%%%%%%%%%%%%%%%%%%%%%%%%%%%%%%%%%%%%%%%%%%%%%%%%%%%
%       REFERENCIAS
%%%%%%%%%%%%%%%%%%%%%%%%%%%%%%%%%%%%%%%%%%%%%%%%%%%%%%%%%%%%%%%%%%%
\newpage
\section{Bibliografía}
\nocite{*}
\bibliographystyle{plain}
\bibliography{referencias.bib}
\end{document}
