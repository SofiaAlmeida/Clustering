%%%
% Plantilla de Memoria
% Modificación de una plantilla de Latex de Nicolas Diaz para adaptarla 
% al castellano y a las necesidades de escribir informática y matemáticas.
%
% Editada por: Mario Román
%
% License:
% CC BY-NC-SA 3.0 (http://creativecommons.org/licenses/by-nc-sa/3.0/)
%%%

%%%%%%%%%%%%%%%%%%%%%%%%%%%%%%%%%%%%%%%%%
% Thin Sectioned Essay
% LaTeX Template
% Version 1.0 (3/8/13)
%
% This template has been downloaded from:
% http://www.LaTeXTemplates.com
%
% Original Author:
% Nicolas Diaz (nsdiaz@uc.cl) with extensive modifications by:
% Vel (vel@latextemplates.com)
%
% License:
% CC BY-NC-SA 3.0 (http://creativecommons.org/licenses/by-nc-sa/3.0/)
%
%%%%%%%%%%%%%%%%%%%%%%%%%%%%%%%%%%%%%%%%%

%----------------------------------------------------------------------------------------
%	PAQUETES Y CONFIGURACIÓN DEL DOCUMENTO
%----------------------------------------------------------------------------------------

%%% Configuración del papel.
% microtype: Tipografía.
% mathpazo: Usa la fuente Palatino.
\documentclass[a4paper, 20pt]{article}
\usepackage[a4paper,margin=1in]{geometry}
\usepackage[protrusion=true,expansion=true]{microtype}
\usepackage{mathpazo}

% Indentación de párrafos para Palatino
\setlength{\parindent}{0pt}
  \parskip=8pt
\linespread{1.05} % Change line spacing here, Palatino benefits from a slight increase by default


%%% Castellano.
% noquoting: Permite uso de comillas no españolas.
% lcroman: Permite la enumeración con numerales romanos en minúscula.
% fontenc: Usa la fuente completa para que pueda copiarse correctamente del pdf.
\usepackage[spanish,es-noquoting,es-lcroman,es-tabla,,es-nodecimaldot]{babel}
\usepackage[utf8]{inputenc}
\usepackage{fontenc}
\selectlanguage{spanish}

%%% Matemáticas
\usepackage{amsmath}

%%% Gráficos
\usepackage{graphicx} % Required for including pictures
\usepackage{wrapfig} % Allows in-line images
\graphicspath{{./fig/}}
\usepackage[usexcolor=false, inkscape=true]{svg} % Required for including svg
\svgpath{{./fig/}}
\usepackage[usenames,dvipsnames]{color} % Coloring code



%%% Pseudocódigo
\usepackage{algorithmicx}
\usepackage[ruled]{algorithm}
\usepackage{algpseudocode}

\newcommand{\alg}{\texttt{algorithmicx}}
\newcommand{\old}{\texttt{algorithmic}}
\newcommand{\euk}{Euclid}
\newcommand\ASTART{\bigskip\noindent\begin{minipage}[b]{0.5\linewidth}}
\newcommand\ACONTINUE{\end{minipage}\begin{minipage}[b]{0.5\linewidth}}
\newcommand\AENDSKIP{\end{minipage}\bigskip}
\newcommand\AEND{\end{minipage}}

%%% Código
\usepackage{listings}

%%% Tablas
\usepackage{tabularx}
\usepackage{float}
\usepackage{adjustbox}
\usepackage{booktabs}
\usepackage{array}
\newcolumntype{L}[1]{>{\raggedright\let\newline\\\arraybackslash\hspace{0pt}}m{#1}}
\newcolumntype{C}[1]{>{\centering\let\newline\\\arraybackslash\hspace{0pt}}m{#1}}
\newcolumntype{R}[1]{>{\raggedleft\let\newline\\\arraybackslash\hspace{0pt}}m{#1}}

% Enlaces y colores
\usepackage{hyperref}
\usepackage{xcolor}
\definecolor{webgreen}{rgb}{0,0.5,0}
\hypersetup{
  colorlinks=true,
  citecolor=RoyalBlue,
  urlcolor=RoyalBlue,
  linkcolor=RoyalBlue
}

%%% Bibliografía
%\usepackage[backend=biber]{biblatex}
%\DefineBibliographyStrings{spanish}{
%  urlseen = {Último acceso}
%}
%\addbibresource{IN-P2.bib}

%----------------------------------------------------------------------------------------
%	TÍTULO
%----------------------------------------------------------------------------------------
% Configuraciones para el título.
% El título no debe editarse aquí.
\renewcommand{\maketitle}{
  \begin{flushright} % Right align
  
  {\LARGE\@title} % Increase the font size of the title
  
  \vspace{50pt} % Some vertical space between the title and author name
  
  {\large\@author} % Author name
  \\\@date % Date
  \vspace{40pt} % Some vertical space between the author block and abstract
  \end{flushright}
}

%% Título
\title{\textbf{Título}\\ % Title
Subtítulo} % Subtitle

\author{\textsc{Autor1,\\Autor2} % Author
\\{\textit{Universidad de Granada}}} % Institution

\date{\today} % Date

%-----------------------------------------------------------------------------------------
%	DOCUMENTO
%-----------------------------------------------------------------------------------------

\begin{document}

%-----------------------------------------------------------------------------------------
%	TITLE PAGE
%-----------------------------------------------------------------------------------------

\begin{titlepage} % Suppresses displaying the page number on the title page and the subsequent page counts as page 1
	
	\raggedleft % Right align the title page
	
	\rule{1pt}{\textheight} % Vertical line
	\hspace{0.05\textwidth} % Whitespace between the vertical line and title page text
	\parbox[b]{0.8\textwidth}{ % Paragraph box for holding the title page text, adjust the width to move the title page left or right on the page
		
		{\Huge\bfseries \textit{Clustering}}\\[2\baselineskip] % Title
		{\large\textit{Curso 2019/2020}}\\[4\baselineskip] % Subtitle or further description
		{\Large\textsc{Sofía Almeida Bruno}\\\textsc{Daniel Bolaños Martínez}\\\textsc{José María Borrás Serrano}\\\textsc{Fernando de la Hoz Moreno}\\\textsc{Pedro Manuel Flores Crespo}\\\textsc{María Victoria Granados Pozo}} % Author name, lower case for consistent small caps
		
		\vspace{0.4\textheight} % Whitespace between the title block and the publisher
		
		{\noindent }\\[\baselineskip] % Publisher and logo
	}

\end{titlepage}

%% Resumen (Descomentar para usarlo)
%\renewcommand{\abstractname}{Resumen} % Uncomment to change the name of the abstract to something else
%\begin{abstract}
% Resumen aquí
%\end{abstract}

%% Palabras clave
%\hspace*{3,6mm}\textit{Keywords:} lorem , ipsum , dolor , sit amet , lectus % Keywords
%\vspace{30pt} % Some vertical space between the abstract and first section


%% Índice
{\parskip=2pt
  \tableofcontents
}
\pagebreak
% TODO:
% - Pasar a textit: clustering, clusters, cluster
% TODO:
% - Referenciar referencias...

\section{Introducción}

El clustering consiste en agrupar objetos similares. Dos objetos se consideran similares si, considerando alguna medida de error, las observaciones podrían ser del mismo objeto. Esta clasificación ocurre constantemente en nuestra vida diaria, damos el mismo nombre a objetos que difieren en detalles insignificantes. 

%Las técnicas de clustering se desarrollaron por primera vez en el ámbito aplicado de la taxonomía o clasificación biológica

Si tenemos una serie de observaciones sin clasificar, el objetivo del clustering es agrupar los datos en clases o clusters. Por ejemplo, cuando en ámbitos biológicos se quiere determinar las especies de una planta concreta. Este tipo de problema aparece cuando queremos no solo identificar especies nuevas, sino también cuando queremos establecer las relaciones entre ellas. El término clustering se considera sinónimo a taxonomía numérica o clasificación. En el ámbito de la ciencia de datos se considera problemas diferentes clasificación y agrupamiento. En el primero conocemos de antemano las clases de los objetos, mientras que en el segundo, a partir de los grupos creados se inferirán las características principales de los grupos.

Utilizando el lenguaje matemático, podemos definir el problema como sigue:

\begin{quote}
  Dadas \textbf{$x_1$}$,\dots, $\textbf{$x_n$} medidas de $p$ variables en $n$ objetos considerados \textit{heterogéneos}. El objetivo del análisis cluster es agrupar estos objetos en $k$ clases \textit{homogéneas}, donde $k$ es también desconocido (aunque habitualmente se asume que es mucho menor que $n$).
\end{quote}

Decimos que un grupo es \textit{homogéneo} si sus miembros están cerca unos de otros pero los miembros de otros grupos son muy diferentes a estos. Esto lleva a definir dos métricas entre los puntos para indicar el grado de alejamiento y el de asociación o similaridad. Se pueden tomar distintas distancias, creando aproximaciones diferentes al problema (trataremos este tema en más profundidad en la Sección \ref{sec:medidas}).

El análisis cluster se aplica en numerosos campos como las ciencias naturales, médicas, económicas, \textit{marketing}, \dots En \textit{marketing}, por ejemplo, es útil dividir a los clientes y conocer las necesidades de cada segmento de mercado para lograr alcanzar a los clientes potenciales. En psicología puede ser útil encontrar tipos de personalidad a partir de los cuestionarios realizados. En arqueología se puede aplicar esta técnica para clasificar objetos en diferentes periodos.

Para llevar a cabo un análisis cluster hay que realizar principalmente dos pasos:

\begin{enumerate}
\item Elegir una medida de proximidad.
\item Elegir un algoritmo para construir los grupos. Hay dos grupos principales: de particionamiento y jerárquicos (que a su vez se dividen entre divisivos y aglomerativos)
\end{enumerate}

\section{Medidas}\label{sec:medidas}
\section{Métodos de agrupamiento}
\subsection{Jerárquicos}
\subsection{No jerárquicos}
\section{Número de clústeres}
\section{Parte práctica}
%%%%%%%%%%%%%%%%%%%%%%%%%%%%%%%%%%%%%%%%%%%%%%%%%%%%%%%%%%%%%%%%%%%
%       REFERENCIAS
%%%%%%%%%%%%%%%%%%%%%%%%%%%%%%%%%%%%%%%%%%%%%%%%%%%%%%%%%%%%%%%%%%%
\newpage
\section{Bibliografía}
\nocite{*}
\bibliographystyle{plain}
\bibliography{referencias.bib}
\end{document}
